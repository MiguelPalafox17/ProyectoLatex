\documentclass{article}
\usepackage{graphicx} % Required for inserting images
\usepackage{amsmath,mathtools}
\usepackage{float}
\usepackage{ragged2e}
\usepackage[none]{hyphenat}

\usepackage{newtxtext, newtxmath}

\usepackage[spanish]{babel}
\usepackage[utf8]{inputenc}
\usepackage[backend=biber]{biblatex}
\bibliography{referencias}

\usepackage{tikz}
\usetikzlibrary{mindmap}

\usepackage{circuitikz}
\ctikzset{bipoles/thickness=1.2}
\usetikzlibrary{shapes,arrows}

\usepackage{pgfplots}
\pgfplotsset{width=10cm,compat=newest}

\usepackage[
paperwidth=6in,
  paperheight=9in,
  top=0.45in,
  bottom=0.90in,
  left=1.0in,
  right=0.80in,
  headheight=13.6pt
]{geometry}


\title{Capitulo 5} 
\author{Estrada Aguayo Emmanuel Roberto y alumnos de 3-4}
\date{28 de Agosto de 2025 }



\begin{document}
\sloppy
\maketitle
\begin{justifying}
    \noindent Hasta ahora no hemos abordado el cambio. Pero, si creemos en la ciencia,
    debemos defender el postulado ontologico de que todo esta en constante cambio. De hecho, 
    las ciencias describen, explican, precide, controlan o provocan cambios de diversos tipos, composición
    el movimiento, la creacion, la division y la evolucion. Por lo tanto, la ontologia deberia analizar y 
    sistematizar estos diversos tipos de cambio.
\end{justifying}

\begin{tikzpicture}
  \path[mindmap,concept color=black,text=white]
    node[concept] {Palafoxin Science }
    [clockwise from=0]
    child[concept color=green!50!black] {
      node[concept] {practical}
      [clockwise from=90]
      child { node[concept] {algorithms} }
      child { node[concept] {data structures} }
      child { node[concept] {pro\-gramming languages} }
      child { node[concept] {software engineer\-ing} }
    }  
    child[concept color=blue] {
      node[concept] {applied}
      [clockwise from=-30]
      child { node[concept] {databases} }
      child { node[concept] {WWW} }
    }
    child[concept color=red] { node[concept] {technical} }
    child[concept color=orange] { node[concept] {theoretical} };
\end{tikzpicture}

\part*{Parte}

\section*{Como escribir en \LaTeX}

\subsection*{Escritura de texto}

\subsubsection*{Textos matemático}

\subsubsection*{Codificación de imagenes}
\noindent - - -.

\bigskip

\part{Parte}

\section{Como escribir en \LaTeX}

\subsection{Escritura de texto}

\subsubsection{Textos matemático}

\subsubsection{Codificación de imagenes}

\noindent - - -.

\bigskip

Referencia a Wikipedia ~\cite{eswiki:166659213}.



\printbibliography


\end{document}
